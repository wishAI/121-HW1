\item (9 marks) One way to better understand a computational system is to look at the minimum set of primitives (simple operations) that are sufficient to express all the tasks performed by the system. In  this question, you will consider a (mythical)  ANDNOT gate, and prove that every  truth table  can be implemented  using \textbf {only}  ANDNOT gates  and the constant  \texttt  {True}. We  define  the  operation of  the  ANDNOT  gate as  follows: $x \textrm{ANDNOT} y$  is the same as  $x \land \lnot y$.
\begin{enumerate}
\item (3 marks) Show that $\lnot$ can be simulated using only ANDNOT and \texttt {True}. That is, design a logical expression that uses only ANDNOT and \texttt {True} and that, given a single proposition $p$, is logically equivalent to $\lnot p$. You must prove that your logical expression is correct using equivalence rules. (See, e.g., Epp 4ed page 35, or Dave's excellent formula sheet)
% You may use tikz to draw fancy latex gates, or you can
% create them via logisim, screenshot the result, 
% upload it to this project, and then use the 
% "includegraphics" command to place it in this document.
% If you decide to use Logisim, use any of the gates available
% as the ANDNOT gate (you are not allowed to use any other
% gates, so there will be no confusion possible).
\vspace{4in}
\item (3 marks) Show that $\land$ can be simulated using only ANDNOT and \texttt {True}. That is, design a logical expression that uses only ANDNOT and \texttt {True} and that, given propositions $p$ and $q$, is logically equivalent to $p \land q$. You must prove that your logical expression is correct using equivalence rules. (See, e.g., Epp 4ed page 35, or Dave's excellent formula sheet)
\vspace{4in}
\item (3 marks)  Show that $\lor$ can be simulated using only ANDNOT and \texttt {True}. That is, design a logical expression that uses only ANDNOT and \texttt {True} and that, given propositions $p$ and $q$, is logically equivalent to $p \lor q$. You must prove that your logical expression is correct using equivalence rules. (See, e.g., Epp 4ed page 35, or Dave's excellent formula sheet)
\vspace{4in}
\end{enumerate}
Since for every truth table over $k$ atomic propositions, we can write a propositional form that matches the truth table using $\lnot$, $\land$ and $\lor$, your answers to parts~(a), (b) and~(c) show that you every propositional form is logically equivalent to a propositional form that uses only ANDNOT and \texttt {True}.

Addendum: when we use a computational system, such as a circuit built out of gates, or (later in the course) regular expressions, we like it to have as many features as possible since it makes it more convenient for us (think of how painful it would be to design circuits using only ANDNOT and \texttt {True}. Or using only NAND gates!) On the other hand, when we want to reason about what this system can or can not do, we like to use the smallest possible set of features, because it's easier to think about a small set of features than a large one (there are fewer cases to consider). That's why proving that every circuit can be designed using only ANDNOT and \texttt {True} can be useful.